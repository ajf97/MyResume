%!TEX TS-program = xelatex
%!TEX encoding = UTF-8 Unicode
% Awesome CV LaTeX Template for Cover Letter
%
% This template has been downloaded from:
% https://github.com/posquit0/Awesome-CV
%
% Authors:
% Claud D. Park <posquit0.bj@gmail.com>
% Lars Richter <mail@ayeks.de>
%
% Template license:
% CC BY-SA 4.0 (https://creativecommons.org/licenses/by-sa/4.0/)
%


%-------------------------------------------------------------------------------
% CONFIGURATIONS
%-------------------------------------------------------------------------------
% A4 paper size by default, use 'letterpaper' for US letter
\documentclass[11pt, a4paper]{awesome-cv}

% Configure page margins with geometry
\geometry{left=1.4cm, top=1.2cm, right=1.4cm, bottom=.8cm}

% Specify the location of the included fonts
\fontdir[fonts/]

% Color for highlights
% Awesome Colors: awesome-emerald, awesome-skyblue, awesome-red, awesome-pink, awesome-orange
%                 awesome-nephritis, awesome-concrete, awesome-darknight
\colorlet{awesome}{awesome-red}
% Uncomment if you would like to specify your own color
% \definecolor{awesome}{HTML}{CA63A8}

% Colors for text
% Uncomment if you would like to specify your own color
% \definecolor{darktext}{HTML}{414141}
% \definecolor{text}{HTML}{333333}
% \definecolor{graytext}{HTML}{5D5D5D}
% \definecolor{lighttext}{HTML}{999999}

% Set false if you don't want to highlight section with awesome color
\setbool{acvSectionColorHighlight}{true}

% If you would like to change the social information separator from a pipe (|) to something else
\renewcommand{\acvHeaderSocialSep}{\quad\textbar\quad}


%-------------------------------------------------------------------------------
%	PERSONAL INFORMATION
%	Comment any of the lines below if they are not required
%-------------------------------------------------------------------------------
% Available options: circle|rectangle,edge/noedge,left/right
\photo[circle,noedge,right]{./profile.png}
\name{Charles}{Zhang}
% \position{Software Architect{\enskip\cdotp\enskip}Security Expert}
\address{1600 Grand Avenue, Saint Paul, MN}

\mobile{(+1) 612-859-0081}
\email{zzhang4@macalester.edu}
\homepage{zcczhang.github.io}
\github{zcczhang}
\linkedin{charleszzz}
% \gitlab{gitlab-id}
% \stackoverflow{SO-id}{SO-name}
% \twitter{@twit}
% \skype{skype-id}
% \reddit{reddit-id}
% \medium{madium-id}
% \googlescholar{googlescholar-id}{name-to-display}
%% \firstname and \lastname will be used
% \googlescholar{googlescholar-id}{}
% \extrainfo{extra informations}

\quote{``Be the change that you want to see in the world."}


%-------------------------------------------------------------------------------
%	LETTER INFORMATION
%	All of the below lines must be filled out
%-------------------------------------------------------------------------------
% The company being applied to
\recipient
  {Feburary 05, 2020\\\\Lori B. Ziegelmeier}
  {Professor\\Mathematics,Statistics, and Computer Science Department\\Macalester College\\Saint Paul, MN 55105}
% The date on the letter, default is the date of compilation
\letterdate{\today}
% The title of the letter
%\lettertitle{Application for Research: Adaptive Robot Navigation: Building Robots That Learn}

% How the letter is opened
\letteropening{\textbf{Dear Professor Ziegelmeier,}}
% How the letter is closed
\letterclosing{Sincerely,}
% Any enclosures with the letter
% \letterenclosure[Attached]{Curriculum Vitae}


%-------------------------------------------------------------------------------
\begin{document}

% Print the header with above personal information
% Give optional argument to change alignment(C: center, L: left, R: right)
\makecvheader[L]

% Print the footer with 3 arguments(<left>, <center>, <right>)
% Leave any of these blank if they are not needed
% \makecvfooter
%   {\today}
%   {Charles Zhang~~~·~~~Cover Letter}
%   {}

% Print the title with above letter information
\makelettertitle

%-------------------------------------------------------------------------------
%	LETTER CONTENT
%-------------------------------------------------------------------------------
\begin{cvletter}
\hyphenpenalty=5000
\tolerance=1000

I was thrilled to see your listing for the research about Topological Data Analysis (TDA) for Summer 2020. As a highly passionate Mathematics and Computer Science student at Macalester College with a solid background in pure math, programming, and data analysis, I am confident that I would be a strong addition to your team.

I love unveiling the elegance of mathematics, especially geometry. In high school, I won the first prize in the Chinese Mathematics League. At that time, I was indulged in a problem involved with trajectory in plane geometry, so I generalized that problem and finally published an article about the geometric transformation and trajectory in \textit{Chinese Periodical Mathematical Study and Research}. Gradually, I was exposed to the algebraic geometry, discrete geometry, and topology, and feeling the power of the geometry. Since I am also intrigued by the application of mathematics, I love learning data science and computer science so that I can solve more complex and practical problems by computer as well, but I never thought about the combination of the topology and data analysis. As I always describe the mathematics as an "elegant" language, I believe I can open up a new world to analyze data by more efficient and ingenious methods after I learn TDA, improving my understanding of the geometry especially the topology and providing me experience for further study in graduate school in the future. 

I am a self-motivated and fast learner. Though I am a first-year student, I love challenging and learning extra knowledge that I am interested in by myself. In the winter break, I learned machine learning and I won $13^{th}$ place out of 19526 teams and individuals(0.06\%) in Kaggle Data Science competition building models in Python. As I looked through the winners algorithms, to my surprise, the algorithm involved with TDA is much faster and more accurate than traditional methods while analyzing high dimensions and clustered data. Therefore, I wish eagerly that I can have this opportunity working with your research to learn how to employ topology in data analysis and visualization.

 For your convenience, I have attached a resume that further outlines my relevant skill set, experience, and accomplishments. Thank you for your consideration, and I look forward to hearing from you soon!


\end{cvletter}


%-------------------------------------------------------------------------------
% Print the signature and enclosures with above letter information
\makeletterclosing

\end{document}
